\documentclass[12pt]{article}
\usepackage[utf8]{inputenc}
\usepackage{amsmath,amssymb,hyperref,array,xcolor,multicol,verbatim,mathpazo,algorithm,algpseudocode,enumerate,tikz}
\usepackage[normalem]{ulem}
\usepackage{graphicx}

\newenvironment{problem}[2][Problem]{\begin{trivlist}
\item[\hskip \labelsep {\bfseries #1}\hskip \labelsep {\bfseries #2.}]}{\end{trivlist}}

\begin{document}

%%%% In most cases you won't need to edit anything above this line %%%%

\title{\vspace{-4cm}CS 270 Homework 6}
\author{Neel Gupta} 
\maketitle

\begin{problem}{1}
    Now let us consider the knapsack problem where you have infinitely many items of each kind. Namely, there are $n$ different types of items. All the items of the same type $i$ have equal size $w_i$ and value $v_i$. You are offered with infinitely many items of each type. Design a dynamic programming algorithm to compute the optimal value you can get from a knapsack with capacity $W$.
    \begin{enumerate}[a.]
        \item Define subproblems to be solved.
        \item Write a recurrence relation for the subproblems.
        \item Make sure you specify
        \begin{enumerate}[i.]
            \item base cases and their values
            \item where the final answer can be found
        \end{enumerate}
        \item What is the complexity of your solution?
    \end{enumerate}
\end{problem}
\textit{Answer:}

a. For a given capacity $W$, find the maximum value that can be obtained through choosing some various items amongst the first $i$ items.

b. Let $mem$ be an array for conducting dynamic programming on. Then $mem[i][w]$ will be the maximum capacity of the first $i$ items with a knapsack capacity of $w$. At every level, we have two choices.
\begin{itemize}
    \item Do not include any items of type $i$, then $mem[i][w]=mem[i-1][w]$
    \item Include 1 or more items of type $i$, then $mem[i][w]=$ max($mem[i][w],mem[i][w-k*w_i]+k*v_i$) where $k$ is the number of items of type $i$ that are incluided in the knapsack.
\end{itemize}
Therefore, the recurrence relation can be expressed as 
$$
mem[i][w]= \text{max}(mem[i-1][w], mem[i][w-k*w_i]+k*v_i)\text{when }k*w_i\leq W 
$$
\newpage
c. \begin{algorithmic}
    \Procedure{infiniteKnapsack}{$w,v,W$}
        \State Intitialize $mem[0][w] \text{ } \forall w = 0,1,...,W$
        \For{$i=1,2,...,n$}
            \For{$w=0,1,...,W$}
                \State $mem[i][w]=mem[i-1][w]$
                \For{$k=1,2,...,\lfloor \frac{w}{w_i}\rfloor$}
                    \State $mem[i][w]=\text{max}(mem[i][w], mem[i][w-k*w_i]+k*v_i)$
                \EndFor
            \EndFor
        \EndFor
        \State return $mem[n][W]$
    \EndProcedure
\end{algorithmic}
The base case is that none of each item has no value. The answer can be found in $mem[n][W]$.

d. The time complexity of the algorithm is $\Theta(nW^2)$, where $n$ is the number of types of items and $W$ is the capacity of the knapsack. For each of the $n$ iterations, the first inner loop does $W$ iterations and the second inner loop does upto $\frac{W}{w_i}$ iterations. Thus, the total runtime is $\Theta(nW^2)$. The total space complexity of this algorithm is $\Theta(nW)$ for the DP table.

\begin{problem}{2}
    Suppose we want to replicate a file over a collection of $n$ servers, labeled $s_1, s_2, ..., s_n$. To place a copy of the file at server $s_i$ results in a \textit{placement cost} $c_i$, for an integer $c_i>0$.
    
    Now, if a user requests the file from seriver $s_i$, and no copy of the ile is present at $s_i$, then the servers $s_{i+1}, s_{i+2}, s_{i+3}, ...$ are searched in order until a copy of the file is finally found, say at server $s_j$, where $j>i$. This results in an \textit{access cost} of $j-i$. The access cost is 0 if $s_i$ holds a copy of the file. We will require that a copy of the file be placed at server $s_n$, so that all such searches will terminate, at the latest, at $s_n$.

    We'd like to place copies of the files at the servers so as to minimize the sum of placement and access costs. Formally, we say that a \textit{configuration} is a choise, for each server $s_i$ with $i=1,2,...,n-1$of whether to place a copy of the file at $s_i$ or not. (Recall that a copy is always placed at $s_n$.) The \textit{total cost} of a configuration is the sum of all placement costs for servers with a copy of the file, plus the sum of all acccess costs associated with all $n$ servers.

    \begin{enumerate}[a.]
        \item Define subproblems to be solved.
        \item Write a recurrence relation for the subproblems.
        \item Make sure you specify
        \begin{enumerate}[i.]
            \item base cases and their values
            \item where the final answer can be found
        \end{enumerate}
        \item What is the complexity of your solution?
    \end{enumerate}
\end{problem}
\textit{Answer:}

a. The subproblems would be to find the minimum cost solution to get the file to a server $i$.

b. Consider an optimal solution $O_i$ which has the minimum cost for getting the file to server $i$. To get to server $j$ from server $i$ given that $j>i$, we must pay the access cost $c_j$ as well as all the access costs for servers $i+1$ to $j-1$.
\begin{align*}
    O_j &= c_j + O_i + {(j-i-1) + (j-i-2) + ... + 1}\\
    O_j &= c_j + O_i + \frac{(j-i)(j-i-1)}{2}
\end{align*}
Therefore, a recurrence relation for the problem would be:
$$
T[j]=c_j+\text{min}(\frac{(j-i)(j-i-1)}{2}+T[i]) \text{ } \forall j=i+1,i+2,...,n
$$
c.
\begin{algorithmic}
    \Procedure{replicateFile}{$costs$,$n$}
        \State $T[0] \gets 0$
        \For{$i=1,2,...,n$}
            \For{$j=i+1, i+2, ..., n$}
            \State $T[j]=c_j+\text{min}(\frac{(j-i)(j-i-1)}{2}+T[i])$
            \EndFor
        \EndFor
        \State return $T[n]$
    \EndProcedure
\end{algorithmic}
The base case is that there is no access placement cost to access the first server that the file begins in. The final answer will be in $T[n]$.

d. The total runtime is $\Theta(n^2)$ since there are upto $n$ iterations at each iteration. The total space complexity is $\Theta(n)$ for the table.
\begin{problem}{3}
    Given $n$ balloons, indexed from 0 to $n-1$. Each balloon is painted with a number on it represented by array nums. You are asked to burst all the balloons. If you burst balloon $i$, you will get $nums[left]*nums[i]*nums[right]$ coins where $left$ and $right$ are adjacent indices of $i$. After bursting the balloon, the $left$ and $right$ then become adjacent. You may assume $nums[-1]=nums[n]=1$ and they are not real therefore you can't burst them. Design a dynamic programming algorithm to find the maximum coins you can collect by bursting the balloons wisely.
    
    \begin{enumerate}[a.]
        \item Define subproblems to be solved.
        \item Write a recurrence relation for the subproblems.
        \item Make sure you specify
        \begin{enumerate}[i.]
            \item base cases and their values
            \item where the final answer can be found
        \end{enumerate}
        \item What is the complexity of your solution?
    \end{enumerate}
\end{problem}

\textit{Answer: }

a. The subproblems involves bursting a subset of balloons such that the maximum amount of coins can be collected.

b. Consider the following example where $n=4, nums=[3,1,5,8]$. At every level, we want to define a subproblem that increases the subset of the balloons popped, so we can store the value of the maximum coins per range, but since the order of popping is what matters, we will store the last popped balloon as well as the value within our table.

Let rows represent the starting index $i$ and columns represent the ending index $j$. When $i=j$, the diagonals of the matrix will be filled with the value of beginning the popping at that index and will store that index as well as the last, in this case first as well, popping. Consider the following table after going through the diagonals and bursting that one first with the example.
\begin{center}
    \begin{tabular}{ |c|c|c|c|c| } 
      \hline
         & 0 & 1 & 2 & 3 \\ 
      \hline
      0 & (3,0) & -1 & -1& -1\\ 
      \hline
      1 & -1 & (15,1) & -1 & -1\\ 
      \hline
      2 & -1 & -1 & (40,2) & -1\\ 
      \hline
      3 & -1 & -1 & -1 & (40,3)\\ 
      \hline
    \end{tabular}
\end{center}
Then to find all the subsets of length two we will have to use the values found before and then iterate through all the possible values in the new subset of size $length+1$ to find which one optimally pops ballons in the best order. When $i=0,j=1$, setting $k$ to 0 would mean popping that one last so popping the balloon at index 1 would mean the value at index (1,1) added to what would happen when popping the 0-index ballon last given that the 1-index balloon has already been popped. So, $k=0$, has total value $15+1*3*5 = 30$. Doing something similar for $k=1$, the total value is 3 (from the table at index 0, 0) + 15 = 18, so popping the 0-index balloon after the 1-index balloon in an array of length two is more optimal, therefore we will store 30 along with the best last balloon bursting (30,0) at index (0,1). After doing this same task for all subarrays of length two, we get the following table.
\begin{center}
    \begin{tabular}{ |c|c|c|c|c| } 
      \hline
         & 0 & 1 & 2 & 3 \\ 
      \hline
      0 & (3,0) & (30,0) & -1& -1\\ 
      \hline
      1 & -1 & (15,1) & (135,2) & -1\\ 
      \hline
      2 & -1 & -1 & (40,2) & (48,3)\\ 
      \hline
      3 & -1 & -1 & -1 & (40,3)\\ 
      \hline
    \end{tabular}
\end{center}
To fill in the value at the index (0,2), we will start with $k=0$ meaning the 0-index is popped last, therefore we use the table to lookup what the max value that was attained from a length of size 2 problem from index 1 to 2, which is exactly the value stored at index (1,2), thereofre the value at (0,2) = value at (1,2) + $1*3*8 = 159$ and ending at index 0, we will do this as well for $k=1,2$ to see that 0 being the last balloon to burst in this group maximizes the value. Thus, we store (159,0) at index (0,2). Repeating this for index (1,3), we see that when popping the 3-index balloon last we also get 159.
\begin{center}
    \begin{tabular}{ |c|c|c|c|c| } 
      \hline
         & 0 & 1 & 2 & 3 \\ 
      \hline
      0 & (3,0) & (30,0) & (159,0)& -1\\ 
      \hline
      1 & -1 & (15,1) & (135,2) & (159,3)\\ 
      \hline
      2 & -1 & -1 & (40,2) & (48,3)\\ 
      \hline
      3 & -1 & -1 & -1 & (40,3)\\ 
      \hline
    \end{tabular}
\end{center}
For the last subproblem of length 4, for $k=0,1,2,3$, we get the values of of 162, 52, 57, and 167. Therefore, we store (167,3) at index (0,3).  
\begin{center}
    \begin{tabular}{ |c|c|c|c|c| } 
      \hline
         & 0 & 1 & 2 & 3 \\ 
      \hline
      0 & (3,0) & (30,0) & (159,0)& (167,3)\\ 
      \hline
      1 & -1 & (15,1) & (135,2) & (159,3)\\ 
      \hline
      2 & -1 & -1 & (40,2) & (48,3)\\ 
      \hline
      3 & -1 & -1 & -1 & (40,3)\\ 
      \hline
    \end{tabular}
\end{center}
We can then backtrack on the indices ending with 3, then checking the value at (0,2) to see that it ends in 0, then seeing the value of (1,2) to see that it ends at index 2, and then finally left with index 1. Therefore, the correct order of popping would be 1, 2, 0, 3 to maximize profit.

Therefore, we can write the following recurrence relation where $mem$ will be our DP table. 
$$
mem[i][j] = \text{max}(mem[i][j], mem[i][k-1]+mem[k+1][i] + nums[i-1]*nums[k]*nums[j+1])
$$
$mem[i][k-1]$ is the value before the k-index balloon is popped. $mem[k+1][i]$ is the value after the k-index balloon is popped. $nums[i-1]$ and $nums[j+1]$ are the numbers just outside of the range which will multiply with the k-index ballon after the before and after balloons are all popped.

c. 
\begin{algorithmic}
    \Procedure{maxBalloonPoppingCoins}{$nums$}
        \State $n\gets$ len($nums$)
        \State Intitialize $mem$ to matrix of size $n x n$
        \For{$len = 1$ to $n$}
            \For{$i=1 $ to $n-len$}
                \State $j \gets i+len-1$
                \For{$k=i$ to $j$}
                    \If{$i=0$} $left \gets 1$
                    \Else  \text{ } $left \gets nums[i-1]$
                    \EndIf
                    \If{$j=n-1$} $right \gets 1$
                    \Else \text{ }$right \gets nums[j+1]$
                    \EndIf
                    \State $mem[i][j] = \text{max}(mem[i][j],$ \\ $ mem[i][k-1]+mem[k+1][i] + left*nums[k]*right)$
                \EndFor
            \EndFor
        \EndFor
        \State return $mem[0][n-1]$
    \EndProcedure
\end{algorithmic}
For the base case of only having 1 balloon, the following algorithm would still work since left and right would be 1, so the return value of would be in $mem[0][0]$.

d. The overall runtime is $\Theta(n^3)$ since we can have up to $n$ iterations for each subset and there are $n^2$ of them. The overall space complexity is $\Theta(n^2)$ for the DP table.
\begin{problem}{4}
    Suppose you have a rod of length $N$, and you want to cut up the rod and sell the pieces in a way that maximizes the total amount of money you get. A piece of length $i$ is worth $p_i$ dollars. Devise a dynamic programming algorithm to determine the maximum amount of money you can get by cutting the rod strategically and selling the cut pieces.
    
    \begin{enumerate}[a.]
        \item Define subproblems to be solved.
        \item Write a recurrence relation for the subproblems.
        \item Make sure you specify
        \begin{enumerate}[i.]
            \item base cases and their values
            \item where the final answer can be found
        \end{enumerate}
        \item What is the complexity of your solution?
    \end{enumerate}
\end{problem}
\textit{Answer:}

a. Given the prices of each piece, the subproblems can be said to be findng the maximum amount of money that is recieved after cutting a rod of length $i$.

b. The recurrence relation for the problem would be 
$$
rev(i) = \text{max}(price[j] + rev(i-j)), \forall j = 1,2, ..., i
$$
where $price[j]$ is the price of a rod of length $j$ and $rev(i-j)$ is the maximum possible revenue that can be recieved from a rod of size $i-j$.

c. 
\begin{algorithmic}
    \Procedure{maxRevenue}{prices,$n$}
        \State revenue[0] $\gets$ 0
        \For{$i=1,...,n$}
            \State currMax $\gets$ -1
            \For{$j=1,...,i$}
                \State currMax $\gets$ max(currMax, prices[$j$]+revenue[$i-j$])
            \EndFor
            \State revenue[i] $\gets$ currMax
        \EndFor
        \State return revenue[$n$]
    \EndProcedure
\end{algorithmic}

d. The base case for when a rod of length 0 is that the maximum revenue will be 0, so revenue[0] = 0. The final answer can be found in revenue[$n$]. 

e. The time complexity of the solution is $O(n^2)$ since we have $n$ iterations with upto $n$ iterations each. The space complexity of the solution would be $\Theta(n)$.
\begin{problem}{5}
   You are trying to run a large computing job in which you need to simulate a physical system for as many discrete \textit{steps} as you can. The lab you're working in has two large supercomputers (which we'll call $A$ and $B$) which are capable of processing this job. However, you're not one of the priority users of these supercomputers, so at any given point in time, you're only able to use as many spare cycles as these machines have available

   Here's   z the problem you face. Your job can only run on one of the machines machines in any given minute. Over each of the next $n$ minutes, you have a "profile" of how much processing power is available on each machine. In minute $i$, you would be able to run $a_i>0$ steps of the simulation if your job is on machine $A$, and $b_i>0$ steps of the simulation if your job is on machine $B$, You also have the ability move your job from one machine to the other, but doing this costs you a minute of time in which no processing is done on your job.

   So, given a sequence of $n$ minutes, a \textit{plan} is a specied by a choice of $A$, $B$, or \textit{"move"} for each minute, with the property that choices $A$ and $B$ cannot appear in consecutive minutes. The \textit{value} of a plan is the total number of steps that you manage to execute over the $n$ minutes: so it's the sum of $a_i$ over all minutes in which the job is on $A$, plus the sum of all $b_i$ over all minutes in which the job is on $B$.

   Given values $a_1,a_2,...,a_n$ and $b_1,b_2,...,b_n$, find a plan of maximum value. Note that your plan can start with either of the machines $A$ or $B$ in minute 1.
    \begin{enumerate}[a.]
        \item Show that the following algorithm does not correctly solve this problem, by giving an instance on which it does not return the correct answer. In your example, say what the correct answer is and also what the algorithm above finds.
        \begin{algorithmic}
            \State In minute 1, choose the machine with the larger of $a_1, b_1$
            \State $i \gets 2$
            \While{$i\leq n$}
                \State What was the choice in minute $i-1$?
                \If{$A$}
                    \If{$b_{i+1} > a_i+a_{i+1}$}
                        \State Choose $move$ in minute $i$ and $B$ in minute $i+1$
                        \State Proceed to iteration $i+2$.
                    \Else 
                        \State Choose A in minute $i$
                        \State Proceed to iteration $i+1$
                    \EndIf
                \ElsIf{$B$} behave as above with roles of $A$ and $B$ reversed
                \EndIf
            \EndWhile
        \end{algorithmic}
        \item Give an efficient algorithm that takes values $a_1,a_2,...,a_n$ and $b_1,b_2,...,b_n$ and returns the value of an optimal plan.
        \begin{enumerate}[i.]
            \item Define subproblems to be solved.
            \item Write a recurrence relation for the subproblems.
            \item Make sure you specify
            \begin{enumerate}[i.]
                \item base cases and their values
                \item where the final answer can be found
            \end{enumerate}
            \item What is the complexity of your solution?
        \end{enumerate}
    \end{enumerate}
\end{problem}

\textit{Answer:}

a. Consider the following counter example

\begin{center}
    \begin{tabular}{ |c|c|c|c|c| } 
      \hline
      Minutes & 1 & 2 & 3 & 4\\ 
      \hline
      $A$ & 2 & 1 & 1 & 1\\ 
      \hline
      $B$ & 1 & 10 & 1 & 1\\ 
      \hline
    \end{tabular}
\end{center}
The above algorithm would return $a_1,a_2,a_3,a_4$ as the plan with value 5. A more optimal solution would be to take $b_1,b_2,b_3,b_4$ that has value 13.

b.
\begin{enumerate}[i.]
    \item The subproblems are to find a sequence of instructions such that we maximize the productivity of the machines over our workload.
    \item Let $O$ be our optimal solution and that we have $O[A][i-2]$ and $O[A][i-1]$ which are the optimal solutions at time $i-2$ and $i-1$ that end with machine $A$ and $O[B][i-2]$ and $O[B][i-1]$ which are the optimal solutions at time $i-2$ and $i-1$ that end with machine $B$. At every iteration, we have the choice to stay or move. If we move, we lose a turn where we could have been making money. Therefore, we get the following recurrence relations $$mem[A][i]=\text{max}(mem[A][i-1] + a_i, mem[B][i-2])$$
    $$mem[B][i]=\text{max}(mem[B][i-1] + b_i, mem[A][i-2])$$
    \item 
    \begin{algorithmic}
        \Procedure{chooseMachines}{$a$, $b$, $n$}
            \State $mem[A][0], mem[B][0] \gets 0$
            \State $mem[A][1], mem[B][1] \gets a_1,b_1$, respectively
            \For{$i=2,3,...,n$}
            \If{$mem[A][i-1]+a_i<mem[B][i-2]$}
                \State $mem[A][i] \gets mem[B][i-2]$
            \Else
                \State $mem[A][i] \gets mem[A][i-1]+a_i$
            \EndIf
            \If{$mem[B][i-1]+b_i<mem[A][i-2]$}
                \State $mem[B][i] \gets mem[A][i-2]$
            \Else
                \State $mem[B][i] \gets mem[B][i-1]+b_i$
            \EndIf
            \EndFor
            \State return max($mem[A][n],mem[B][n]$)
        \EndProcedure
    \end{algorithmic}
    The basecase is that not starting the jobs means that at minute 0, no progress has been made. Therefore, $mem[A][0], mem[B][0]$ get initalized to 0. If there is only one job, then the algorithm will return the machine with the higher processing power of the two. The final answer can be found in either $mem[A][n]$ or $mem[B][n]$ depending on which machine would be more optimal to end with.
    \item The total runtime of this algorithm would be $\Theta(n)$ since at most constant time work is being done at each iteration. Therefore, the algorithm runs in linear time. The space complexity would be $O(2n)$ for storing all the progress made by the two factories which is $\Theta(n)$.
\end{enumerate}
 

  

\end{document}